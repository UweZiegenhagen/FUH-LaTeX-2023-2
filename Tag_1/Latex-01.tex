\documentclass[12pt,parskip=half]{scrartcl}

\author{Uwe Ziegenhagen}
\title{Mein erstes \LaTeX-Dokument}

\usepackage[ngerman]{babel} % Silbentrennung
\usepackage{blindtext} % Dummy-Text
\usepackage{microtype} % Mikrotypografie-Erweiterungen aktivieren

\usepackage{hyperref} % Links

\usepackage{hyperref} % Link Setup
\hypersetup{
    bookmarks=true,                     % show bookmarks bar
    unicode=false,                      % non - Latin characters in Acrobat’s bookmarks
    pdftoolbar=true,                        % show Acrobat’s toolbar
    pdfmenubar=true,                        % show Acrobat’s menu
    pdffitwindow=false,                 % window fit to page when opened
    pdfstartview={FitH},                    % fits the width of the page to the window
    pdftitle={Mein erstes Dokument},                        % title
    pdfauthor={Uwe Ziegenhagen},                 % author
    pdfsubject={Abschlussarbeit},                   % subject of the document
    pdfcreator={LaTeX},                   % creator of the document
    pdfproducer={pdflatex},             % producer of the document
    pdfkeywords={Uwe, Ziegenhagen},   % list of keywords
    pdfnewwindow=true,                  % links in new window
    colorlinks=true,                        % false: boxed links; true: colored links
    linkcolor=blue,                          % color of internal links
    filecolor=blue,                     % color of file links
    citecolor=blue,                     % color of file links
    urlcolor=blue                        % color of external links
}


\begin{document}
\maketitle

\tableofcontents % Inhaltsverzeichnis


\clearpage

\section{Einführung}
\subsection{Fragestellung}

In Abschnitt \ref{sec:fazit} zeige ich warum.

Wieviel \% brauche ich für 100 \$? % Diverse Sonderzeichen brauchen den Backslash

Es gibt Empfehlungen zu LaTeX.
Am besten pro Satz eine Zeile benutzen.
Dann tut sich LaTeX leichter damit, den Ort eines Fehlers zu kennzeichnen.

Ein neuer Absatz erfordert eine leere Zeile davor

\blindtext[5]

\subsection{Literaturüberblick}
\subsubsection{Deutschsprachige Literatur}

\blindtext[5]

\subsubsection{Internationale Literatur}

\blindtext[5]

\paragraph{Hallo} \blindtext[5]

\paragraph{Welt} \blindtext[5]

\subparagraph{Hello} \blindtext[5]


\section{Untersuchung}

\blindtext[5] 



\section{Fazit}\label{sec:fazit}

\blindtext[5]


\end{document}

