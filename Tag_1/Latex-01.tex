\documentclass[12pt,parskip=half]{scrartcl}

\author{Uwe Ziegenhagen}
\title{Mein erstes \LaTeX-Dokument}

\usepackage[ngerman]{babel} % Silbentrennung
\usepackage{blindtext} % Dummy-Text
\usepackage{microtype} % Mikrotypografie-Erweiterungen aktivieren

\usepackage{graphicx} % Dateitypen: PNG, PDF, JPG

\usepackage{showlabels}
\usepackage{todonotes}


\usepackage{hyperref} % Links, als letztes Paket laden
\usepackage{hyperref} % Link Setup
\hypersetup{
    bookmarks=true,                     % show bookmarks bar
    unicode=false,                      % non - Latin characters in Acrobat’s bookmarks
    pdftoolbar=true,                        % show Acrobat’s toolbar
    pdfmenubar=true,                        % show Acrobat’s menu
    pdffitwindow=false,                 % window fit to page when opened
    pdfstartview={FitH},                    % fits the width of the page to the window
    pdftitle={Mein erstes Dokument},                        % title
    pdfauthor={Uwe Ziegenhagen},                 % author
    pdfsubject={Abschlussarbeit},                   % subject of the document
    pdfcreator={LaTeX},                   % creator of the document
    pdfproducer={pdflatex},             % producer of the document
    pdfkeywords={Uwe, Ziegenhagen},   % list of keywords
    pdfnewwindow=true,                  % links in new window
    colorlinks=true,                        % false: boxed links; true: colored links
    linkcolor=blue,                          % color of internal links
    filecolor=blue,                     % color of file links
    citecolor=blue,                     % color of file links
    urlcolor=blue                        % color of external links
}


\begin{document}
\maketitle

\tableofcontents % Inhaltsverzeichnis

\listoffigures

\listoftodos

\clearpage

\section{Einführung}
\subsection{Fragestellung} \todo{fsadgfdsg}

In Abschnitt \ref{sec:fazit} zeige ich warum, in Abbildung \ref{fig:katze} auf Seite \pageref{fig:katze} zeige ich wieso

Wieviel \% brauche ich für 100 \$? % Diverse Sonderzeichen brauchen den Backslash

Es gibt Empfehlungen zu LaTeX.
Am besten pro Satz eine Zeile benutzen.
Dann tut sich LaTeX leichter damit, den Ort eines Fehlers zu kennzeichnen.

\includegraphics[width=0.75\textwidth]{Bilder/Katze}


\begin{figure}
\centering
\includegraphics[width=0.75\textwidth]{Bilder/Katze}
\caption{Meine freche Katze}\label{fig:katze}
\end{figure}






Ein neuer Absatz erfordert eine leere Zeile davor

\blindtext[5]

\subsection{Literaturüberblick}
\subsubsection{Deutschsprachige Literatur}

\blindtext[5]

\subsubsection{Internationale Literatur}

\blindtext[5]

\paragraph{Hallo} \blindtext[5]

\begin{figure}
\centering
\includegraphics[width=0.75\textwidth]{Bilder/Katze1}
\caption{Meine kuschelige Katze}\label{fig:katze1}
\end{figure}



\paragraph{Welt} \blindtext[5]

\subparagraph{Hello} \blindtext[5]


\section{Untersuchung}

\blindtext[5] 



\section{Fazit}\label{sec:fazit}

\begin{figure}
\centering
\includegraphics[width=0.75\textwidth,angle=45]{Bilder/miau}
\caption{Meine wuschelige Katze}\label{fig:miau}
\end{figure}


\blindtext[5]


\end{document}

