\documentclass{beamer}

\usepackage[ngerman]{babel}
\usepackage{graphicx}
\usepackage{blindtext}

\usetheme{AnnArbor}

\author{Uwe Ziegenhagen}
\title{Meine erste Präsentation}
\subtitle{Köln, den \today}
\institute{Dante e.V.}

\begin{document}
\maketitle

\begin{frame}
\frametitle{Programm}

\tableofcontents

\end{frame}


\section{Einleitung}

\begin{frame}
\frametitle{Titel der Folie}
\framesubtitle{Titel der Folie}


\begin{itemize}
\item<1-> Hallo
\item<2-> Fernuni
\item<3-> Hagen
\item<4-> Dies 
\item<5-> ist eine
\item<6-> Präsentation
\end{itemize}
\end{frame}

\section{Fazit}

\begin{frame}
\frametitle{Titel der Folie}
\framesubtitle{Titel der Folie}

\begin{enumerate}
\item<2> Hallo
\item<1-> Fernuni
\item<-3> Hagen
\item<2> Dies 
\item<1,3> ist eine
\item Präsentation
\end{enumerate}
\end{frame}


\begin{frame}
\frametitle{Meine Katze}

\begin{center}
\includegraphics[width=0.75\textwidth]{Bilder/Katze}
\end{center}

\end{frame}

\begin{frame}
\frametitle{Melli}

\begin{columns}

\begin{column}{0.33\textwidth}
\begin{center}
\includegraphics[width=\textwidth]{Bilder/Katze}
\end{center}
\end{column}
\begin{column}{0.66\textwidth}
\begin{itemize}
	\item {\tiny\blindtext }
\end{itemize}
\end{column}

\end{columns}

\end{frame}

\begin{frame}
\frametitle{Mathe}

\begin{equation}
-\frac{p}{2}\pm \sqrt{\left(\frac{p}{2} \right)^2-q}
\end{equation}

\end{frame}



\end{document}