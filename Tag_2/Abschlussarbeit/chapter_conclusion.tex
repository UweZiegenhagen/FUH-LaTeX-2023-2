% !TeX root = Abschlussarbeit_Ziegenhagen.tex
\chapter{Fazit und Ausblick}

\blindtext[5]

\begin{itemize}
\item Hallo,
\item ich bin
\item eine Liste
\begin{itemize}
\item Hallo,
\item ich bin
\item eine Liste
\begin{itemize}
\item Hallo,
\item ich bin
\item eine Liste
\end{itemize}
\end{itemize}
\end{itemize}

\blindtext\footnote{\blindtext[2]}


\begin{enumerate}
\item Hallo,
\item ich bin
\item eine Liste
\begin{enumerate}
\item Hallo,
\item ich bin
\item eine Liste
\begin{enumerate}
\item Hallo,
\item ich bin
\item eine Liste
\end{enumerate}
\end{enumerate}
\end{enumerate}

\begin{description}
\item[Äpfel] \blindtext
\item[Birnen] \blindtext
\item[Pfirsiche] \blindtext
\end{description}

% paralist Beispiele

\begin{compactitem}
\item Hallo,
\item ich bin
\item eine Liste
\end{compactitem}

\begin{compactenum}
\item Hallo,
\item ich bin
\item eine Liste
\end{compactenum}

\begin{compactdesc}
\item[Äpfel] \blindtext
\item[Birnen] \blindtext
\item[Pfirsiche] \blindtext
\end{compactdesc}

\section{Textauszeichnung}

\begin{tabular}{ll} \toprule
Auszeichnung & Beispiel \\ \midrule
\textbackslash textbf & \textbf{Fetter Text} \\
\textbackslash textit & \textit{Kursiver Text} \\
\textbackslash texttt & \texttt{Monospace Schreibmaschine} \\
\textbackslash textsc & \textsc{Text mit Kapitälchen} \\
\textbackslash textsl & \textsl{Geneigter Text} \\
\textbackslash textbf  \textbackslash textit  & \textit{\bfseries Geneigter Text} \\
\end{tabular}



