\usepackage[ngerman]{babel} % eindeutschen
\usepackage{blindtext} % Dummy-Text
\usepackage{booktabs} % schöne Tabellen
\usepackage{xcolor} % Farben in LaTeX über xcolor

\usepackage{amsmath} % bessere Mathematik
\usepackage{esvect} % \vv Vektor-Befehl
\usepackage[standard,thmmarks]{ntheorem} % Theorem, Proof

\usepackage{paralist} % für kompakte Aufzählungen

\usepackage{palatino}
\usepackage{microtype}
\usepackage{showkeys}
\usepackage{lineno}
\linenumbers

\makeatletter % Mach das @ zu einem normalen Zeichen
\newcommand{\avg}{\mathop{\operator@font avg}}
\makeatother

\newcommand{\person}[1]{\textcolor{orange}{\textsc{#1}}}

%hyperref als letztes Paket laden, außer man weiß genau was man tut
\usepackage{hyperref}
\hypersetup{
    bookmarks=true,                     % show bookmarks bar
    unicode=false,                      % non - Latin characters in Acrobat’s bookmarks
    pdftoolbar=true,                        % show Acrobat’s toolbar
    pdfmenubar=true,                        % show Acrobat’s menu
    pdffitwindow=false,                 % window fit to page when opened
    pdfstartview={FitH},                    % fits the width of the page to the window
    pdftitle={My title},                        % title
    pdfauthor={Author},                 % author
    pdfsubject={Subject},                   % subject of the document
    pdfcreator={Creator},                   % creator of the document
    pdfproducer={Producer},             % producer of the document
    pdfkeywords={keyword1, key2, key3},   % list of keywords
    pdfnewwindow=true,                  % links in new window
    colorlinks=true,                        % false: boxed links; true: colored links
    linkcolor=blue,                          % color of internal links
    filecolor=blue,                     % color of file links
    citecolor=blue,                     % color of file links
    urlcolor=blue                        % color of external links
}