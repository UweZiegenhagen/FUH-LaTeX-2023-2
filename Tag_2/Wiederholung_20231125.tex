\documentclass[parskip=half]{scrartcl}

\author{Uwe Ziegenhagen}
\title{Meine Wiederholung}

\usepackage[ngerman]{babel}

\usepackage{blindtext}
\usepackage{booktabs}

\makeatletter % Mach das @ zu einem normalen Zeichen
\newcommand{\avg}{\mathop{\operator@font avg}}
\makeatother

% Farben in LaTeX über xcolor
\usepackage{xcolor}

\newcommand{\person}[1]{\textcolor{orange}{\textsc{#1}}}

\usepackage[]{amsmath}
\usepackage[]{esvect}
\usepackage[standard,thmmarks]{ntheorem}

\begin{document}
\maketitle

\tableofcontents

\section{a}
\subsection{b}
\subsection{c}

\blindtext[5]


\begin{table}[ht]
\begin{center}
\begin{tabular}{cp{0.7\textwidth}} \toprule
\textbf{Formel} & \textbf{Erläuterung} \\ \midrule
\( \exists[\{a\}]\mathbf{U} [\{b\}]0 \) & Es gibt keinen Pfad, auf dem so lange keine a-Transitionen möglich ist, bis keine b-Transition mehr möglich ist.\\
\end{tabular}
\end{center}
\end{table}

\person{Albert Einstein} sagt: 
Mathematik im Fließtext \( \exists[\{a\}]\mathbf{U} [\{b\}]0 \)  mit runden Klammern 

\[ \exists[\{a\}]\mathbf{U} [\{b\}]0 \]

\begin{equation}
\exists[\{a\}]\mathbf{U} [\{b\}]0
\end{equation}

\begin{equation}
\sin 45 \avg 4568
\end{equation}

\section[Standard-\LaTeX-Mathematik]{Standard-\LaTeX-Mathematik, die ohne externe Pakete einfach so funktioniert}

\begin{eqnarray} % mit Formelnummer
y &=& (a + b)^2 \\
y &=& a^2 + 2ab +b^2
\end{eqnarray}

\begin{eqnarray*} % ohne Formelnummer
y &=& (a + b)^2 \\
y &=& a^2 + 2ab +b^2 
\end{eqnarray*}

\blindtext

\[
\begin{array}{lcr}
y + y + y &=& (a + b)^2 \\
y &=& a^2 + 2ab +b^2 
\end{array}
\]

\[ % LaTeX 
\bordermatrix{
   & 1 & 2 & 3 \cr
1 & 4 & 2 & 156 \cr
2 & 5 & 33 & 56 \cr
3 & 6 & 2 & 88 \cr
}
\]

\section*{Section ohne Eintrag im TOC}

Macht man mit den gesternten Versionen von manchen Befehlen.

\section{Mathematik mit AMSMath}

\begin{align} % mit Nummern 
a &= c \cdot x \\
a &= c \cdot \tanh z + \sum_{i=1}^{1000} t
\end{align}

\begin{align*} % ohne Nummern 
a &= c \cdot x \\
a &= c \cdot \tanh z + \sum_{i=1}^{1000} t
\end{align*}

\begin{alignat}{3}
a &= c \cdot x &= x\times y \leq 567 \\
a &= c \cdot \tanh z + \alpha\omega &= \sum_{i=1}^{1000} t
\end{alignat}

\begin{alignat*}{3}
a &= c \cdot x &= x\times y \leq 567 \\
a &= c \cdot \tanh z + \alpha\omega &= \sum_{i=1}^{1000} t
\end{alignat*}

\[% Version ohne Klammern
\begin{matrix} 
1 & 0 & 0 \\ 
0 & 1 & 0 \\ 
0 & 0 & 1 \\ 
\end{matrix}
\]

\[% mit runden Klammern (parentheses)
\begin{pmatrix} 
1 & 0 & 0 \\ 
0 & 1 & 0 \\ 
0 & 0 & 1 \\ 
\end{pmatrix}
\]

\[% mit eckigen Klammern
\begin{bmatrix} 
1 & 0 & 0 \\ 
0 & 1 & 0 \\ 
0 & 0 & 1 \\ 
\end{bmatrix}
\]

\[% mit geschweiften Klammern
\begin{Bmatrix} 
1 & 0 & 0 \\ 
0 & 1 & 0 \\ 
0 & 0 & 1 \\ 
\end{Bmatrix}
\]


\[% mit senkrechten Strichen 
\begin{vmatrix} 
1 & 0 & 0 \\ 
0 & 1 & 0 \\ 
0 & 0 & 1 \\ 
\end{vmatrix}
\]

\[% mit doppelten senkrechten Strichen 
\det 
\begin{Vmatrix} 
\ddots  & 0 & \vdots \\ 
0 & \cdots & \dots \\ 
0 & 0 & 1 \\ 
\end{Vmatrix}
\text{ist eine Matrix}
\]

\section{Vektoren}

%Standard-Vektorpfeil skaliert nicht mit
\( \vec{a} \times \vec{def} \)

%esvect Paket laden für \vv-Befehl
\( \vv{a} \times \vv{def} \bigtriangleup \Omega  \)

\section{Abstände}

\( ab \) % kein Abstand

\(a\,b \)

\(a\;b\)

\(a\quad b\)

\(a\qquad b\)


\begin{proof}
fsdfsd
\end{proof}

\begin{theorem}
fsdfsd
\end{theorem}

\begin{lemma}
fsdfsd
\end{lemma}

\begin{corollary}
fsdfsd
\end{corollary}


\end{document}